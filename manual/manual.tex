\documentclass[12pt, a4paper]{article}
\usepackage[russian]{babel}
\usepackage{fontspec}
\setmainfont[
  Ligatures=TeX,
  Extension=.otf,
  BoldFont=cmunbx,
  ItalicFont=cmunti,
  BoldItalicFont=cmunbi,
]{cmunrm}
\usepackage{polyglossia}
\setdefaultlanguage{russian}
\setotherlanguage{english}


\usepackage{geometry}

\geometry{
margin=1.5cm
}

\usepackage{indentfirst}

\usepackage{arydshln}
\usepackage[fleqn]{amsmath}
\usepackage{xfrac}
\usepackage{esint}
\usepackage{amssymb}
\usepackage{mathbbol}
\usepackage[T1]{fontenc}
\usepackage{mathtools}
\usepackage{color}
\usepackage{ulem}
\usepackage{tabu}
\usepackage{multirow}
\usepackage{rotating}

\usepackage[outline]{contour}
\contourlength{1.2pt}

\usepackage{tikz}
\usepackage{graphics}
\usepackage{xcolor}

\usepackage[at]{easylist}

\DeclareMathOperator{\sign}{sign}

\begin{document}

\subsection{Warning}

\textbf{NOTE:} Все данные предоставлены лишь в ознакомительном порядке.

В формулах могут быть ошибки.

%-------------------------------------------------------------------------------

\section{Лабораторная работа №1}

\subsection{Задание}
$ \text{Численно решить ОДУ, построив таблицы c шагом h} \in \left\{ 0.1, 0.01, 0.001 \right\} : $

$ y' = 2ty $

$ y(0) = 1 $

$ t \in [0,1] $

\subsection{Методы}

Дано уравнение $\frac{\partial y}{\partial t} = f(t,\ y)$, где $y = y(t) : \mathbb{R}\to \mathbb{R}$, $f : \mathbb{R}\to \mathbb{R}$, $t$ --- свободная переменная. Тогда его можно решить численно с помощью следующих методов.

\textit{Примечание:} эти методы справедлимы и для систем однородных дифференциальных уравнений. Тогда вместо $y$ будет вектор размерности $n$, а функция должна иметь следующий вид: $f: \mathbb{R}^n\to \mathbb{R}^n$.

Например, для данной системы, $y$ и $f$ будут равны:

$$ \left\{
\begin{aligned}
&\frac{\partial x_1}{\partial t} = x_1^2 + x_2^2 - 4 t x_1 x_2  \\
&\frac{\partial x_2}{\partial t} = 3(x_1-t)(x_2+t) 
\end{aligned}
\right.\quad\Rightarrow\quad y = \begin{pmatrix}x_1\\ x_2 \end{pmatrix},\ f(t,\ y) = 
\begin{pmatrix}
x_1^2 + x_2^2 - 4 t x_1 x_2  \\
3(x_1-t)(x_2+t) 
\end{pmatrix} $$

\textit{Обозначения:}

$ t_0 $ --- точка начала.

$ h $ --- шаг.

$t_n = t_0 + hn$

$ y_n \approx y(t_n) $

\subsubsection{Метод Эйлера (явный)}

$ y_{n+1} = y_n + h \cdot f(t_n, y_n) $

\subsubsection{Модифицированнй метод Эйлера}

$ y_{n+1} = y_n + \frac{h}{2} \cdot [f(t_n, y_n) + f(t_{n+1}, y_n + h \cdot f(t_n, y_n))] $

\subsubsection{Метод Рунге-Кутты 4-го порядка}

$ y_{n+1} = y_n + \frac{h}{6} \cdot [k_n^1 + 2k_n^2 + 2k_n^3 + k_n^4] $

$ k_n^1 = f(t_n, y_n) $

$ k_n^2 = f(t_n + \frac{h}{2}, y_n + \frac{h}{2} k_n^1) $

$ k_n^3 = f(t_n + \frac{h}{2}, y_n + \frac{h}{2} k_n^2) $

$ k_n^4 = f(t_n + \frac{h}{2}, y_n + h k_n^3) $

%-------------------------------------------------------------------------------

\section{Лабораторная работа №2}

\subsection{Задание}
$ \text{Проинтегрировать и построить графики координаты x, скорости и давления} $

$ \text{c шагом h} \in \left\{ 0.1, 0.01, 0.001 \right\} : $


\subsection{Формулы}

$ v' = \frac{1}{m}[P_i S_i - P_j S_j - \nu \cdot |V| \cdot sign V] $

$ x' = v $

$ P_i = \frac{q_i - S_i V}{K^{\text{Упр.}}} \> \> \> \> P_j = \frac{S_j V - q_j}{K^{\text{Упр.}}}$


\subsection{Данные и начальные условия}

$\text{Насос постоянного расхода: } q_i = 1 \frac{\text{м}}{\text{c}} = 10^{-3} \frac{\text{м}}{\text{c}}$

$P_j = 10^5\, \text{Па. (атм. давление)}$

$ x(0) = 0$

$ V(0) = 0$

$ P(0) = 10^5\, \text{Па.} $

%-------------------------------------------------------------------------------

\section{Лабораторная работа №3}

\subsection{Задание}

МС - местное сопротивление

ТП - трубопровод

$x_1$, $x_2$ - координаты при которых происходит переключение канала.

Ударник сам переключает свои каналы.


\subsection{Метод работы перегородки}

$ x(t) $

$ x^k, x^{k+1}, x^{k+2}, \dots $

$ x^k < x_\text{окр.} \wedge x^{k+1} > x_\text{окр.} $

\noindent\begin{easylist}
\ListProperties(Hang1=true, Start1=1)
@ Уменьшаем $\sfrac{ht}{2}$.
@ Мы могли попасть справа или слева $x_\text{окр.}$
@ Уменшаем пока $x$ не окажется слева.
@ Остановиться, когда подходим на расстояние $\Delta\delta = 10^{-6} \text{м}$, тогда $x = x_{\text{окр.}}$, $v = 0$.
\end{easylist}


\subsection{Основные формулы:}

\noindent\begin{easylist}
\ListProperties(Hang1=true, Start1=1)
@ $ p' = h \Phi(q_1 - q_j,\ p,\ C,\ C_{cav}) $
@ $ q' = h G(p_i-p_j-P_\alpha(q),\ q) $
@ $
\Phi(q_{ij},\ p,\ c,\ C) = \left\{
\begin{aligned}
&\frac{p^{\left(1+\frac{1}{\gamma}\right)}q_{ij}}{C_{cav}},  & 0 < p \leqslant p_{\text{Атм.}}\ \text{и} \ 0 < q_{ij}, \\
& \frac{q_{ij}}{C}, & \text{иначе}
\end{aligned}
\right.
$
@ $
P_\alpha(q) = \left\{
\begin{aligned}
&rq,  &Re^{mult} \cdot |q| < Re^{crt} \\
&r_\text{\ae}|q|^\text{\ae}\sign(q), & \text{иначе}
\end{aligned}
\right.
$
@ $ G(dp,\ q) = B\sqrt{|dp|}\left(F\sqrt{\frac{dp}{\xi}}^3 - q\right) $
\end{easylist}


\subsection{Константы:}

\noindent\begin{easylist}
\ListProperties(Hang1=true, Start1=1)
@ $ \mathrm{Nu} = \left\{
\begin{aligned}
& 10^{-6},  &\text{вода} \\
& 3.5 \cdot 10^{-5}, &\text{масло}
\end{aligned}
\right. $ - число Нуссельта
@ $ \gamma = 1.4 $
@ $ \rho = \left\{
\begin{aligned}
& 997 \frac{\text{кг}}{\text{м}^3},  &\text{вода} \\
& 905 \frac{\text{кг}}{\text{м}^3}, &\text{масло}
\end{aligned}
\right. $ - плотность жидкости
@ $ E_s = \rho (\rho_{swd})^2 \approx 1260000 $ - модуль жесткости воды.
@ $ C = \frac{V}{E_s} $
@ $ C_{cav} = 10^{\frac5\gamma}\frac{V}{\gamma} $
@ $ d \approx 2 \text{см} $
@ $ \text{\ae} = 1.75 $
@ $ Re^{crt} = 321 $ - число Рейнольдса
@ $ Re^{mult} = \frac{dH}{\mathrm{Nu} \cdot S} $ - число Рейнольдса
@ $ dH = \frac{4S}{\pi d} $
@ $ B = \frac{1}{l\sqrt{2\rho}} $
@ $ F = S\sqrt{2\rho} $
@ $ \xi = 0.5 \cdot 0.035 \cdot 2 \cdot \left(\frac{1}{\mathrm{eps}(1)} - 1\right)^2 $
@ $ \mathrm{eps}(x) = 0.57 + 0.043(1.1 - x) $
@ $ r = \frac{12g\nu l}{(dH)^2 S} $
@ $ r_\text{\ae} = \frac{0.1582 g \nu^{0.25} l}{(dH)^{1.35} S^{1.75}} $
\end{easylist}


\subsection{Обозначения:}

\noindent\begin{easylist}
\ListProperties(Hang1=true, Start1=1)
@ $ V $ - объем камеры. 
@ $ S $ - площадь сечения местного сопротивления.
@ $ C $ - жесткость воды.
@ $ C_{cav} $ - жесткость воды при кавитации.
\end{easylist}

\end{document}